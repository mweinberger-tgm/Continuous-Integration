\documentclass[letterpaper, 12pt]{article}

%%%%%%%%%%%%%%%%%%%%%%%%%%%%%
% DEFINITIONS
% Change those informations
% If you need umlauts you have to escape them, e.g. for an ü you have to write \"u
\gdef\mytitle{Protokoll}
\gdef\mythema{Continuous Integration}

\gdef\mysubject{SEW}
\gdef\mycourse{5BHIT 2015/16}
\gdef\myauthor{Michael Weinberger}

\gdef\myversion{1.0}
\gdef\mybegin{9. Februar 2016}
\gdef\myfinish{\today}

\gdef\mygrade{Note:}
\gdef\myteacher{Betreuer: Dolezal/Vittori}
%
%%%%%%%%%%%%%%%%%%%%%%%%%%%%%

\input special/preamble.tex

\let\tempsection\section
\renewcommand\section[1]{\vspace{-0.3cm}\tempsection{#1}\vspace{-0.3cm}}
\WithSuffix\newcommand\section*[1]{\tempsection*{#1}}

\let\tempsubsection\subsection
\renewcommand\subsection[1]{\vspace{0cm}\tempsubsection{#1}\vspace{0cm}}

\let\tempsubsubsection\subsubsection
\renewcommand\subsubsection[1]{\vspace{0cm}\tempsubsubsection{#1}\vspace{0cm}}

\linespread{0.94}

\lhead{\mysubject}
\chead{}
\rhead{\bfseries\mythema}
\lfoot{\mycourse}
\cfoot{\thepage}
% Creative Commons license BY
% http://creativecommons.org/licenses/?lang=de
\rfoot{\ccby\hspace{2mm}\myauthor}
\renewcommand{\headrulewidth}{0.4pt}
\renewcommand{\footrulewidth}{0.4pt}

\begin{document}
\parindent 0pt
\parskip 6pt

\pagenumbering{Roman} 
\input{special/title}

\clearpage
\thispagestyle{empty}
\tableofcontents

\newpage
\pagenumbering{arabic}
\pagestyle{fancy}

%\vspace{-0.5cm}
\section{Einführung}
\textit{"Continuous Integration is a software development practice where members of a team integrate their work frequently, usually each person integrates at least daily - leading to multiple integrations per day. Each integration is verified by an automated build (including test) to detect integration errors as quickly as possible. Many teams find that this approach leads to significantly reduced integration problems and allows a team to develop cohesive software more rapidly. This article is a quick overview of Continuous Integration summarizing the technique and its current usage." M.Fowler} \\ \\

Schreibe fünf Testfälle für dein CSV-Projekt und lass diese mithilfe von Jenkins automatisch bei jedem Build testen!

\begin{itemize}
	\item Installiere auf deinem Rechner bzw. einer virtuellen Instanz das Continuous Integration System Jenkins
	\item Installiere die notwendigen Plugins für Jenkins (Git Plugin, Violations, Cobertura)
	\item Installiere Nose und Pylint (mithilfe von pip)
	\item Integriere dein CSV-Projekt in Jenkins, indem du es mit Git verbindest
	\item Schreibe fünf Unit Tests für dein CSV-Projekt
	\item Konfiguriere Jenkins so, dass deine Unit Tests automatisch bei jedem Build durchgeführt werden inkl. Berichte über erfolgreiche / fehlgeschlagene Tests und Coverage
	\item Protokolliere deine Vorgehensweise (inkl. Zeitaufwand, Konfiguration, Probleme) und die Ergebnisse (viele Screenshots!)
\end{itemize}

Viel Spaß! :)
\newpage

\section{Durchführung}

\subsection{Jenkins installieren}
Ich habe mich entschieden, Jenkins in einer virtuellen Instanz zu installieren.

\clearpage

\bibliographystyle{unsrt}
\bibliography{Continuous-Integration_Weinb_5BHIT}
\lstlistoflistings
\listoffigures

\end{document}
